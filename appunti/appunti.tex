\documentclass[a4paper]{article}
\usepackage[utf8]{inputenc}
%\usepackage[italian]{babel}
\usepackage{titling}
\usepackage{graphicx}
\usepackage{wrapfig}
\usepackage{float}
\usepackage{amsmath}
\usepackage{listings}
 \usepackage[table,xcdraw]{xcolor}

\newcommand{\subtitle}[1]{%
	\posttitle{%
		\par\end{center}
	\begin{center}\large#1\end{center}
	\vskip0.5em}%
}

\begin{document}

%opening
\title{Radiation Pressure Dominated Region's Instability in Subcritical Regimes $\alpha$-Accretion Disks around Black Holes in Binary Systems}
\subtitle{Appunti}
\author{Riccardo Aurelio Gilardi}
\maketitle

\tableofcontents
\newpage

\section{Accrescimento nei sistemi binari (introduzione)}
	\subsection{Momento angolare}
	\subsection{Modello $\alpha$-disco}
	
		\emph{Io parlerò solo di modelli del tipo "$\alpha$-disk", ma come devo comportarmi rispetto ai modelli precedenti, spesso citati negli articoli, di Pringle \& Reese e Norikov \&Thorne? Sono stati smentiti? Sono stati uniti?}\\
	
		\emph{Data la natura rivoluzionaria di molte delle sue ipotesi, seguirò più o meno un processo storico, parlando prima del modello secondo S\&S e poi, di come questo modello implicasse dei difetti strutturali, che analizzerò nella sezione sull'instabilità, dove parlerò infine delle ipotesi che hanno permesso di superare l'enpasse}

\section{Struttura del disco secondo Shakura \& Sunyaev}

	\emph{Devo spiegare le equazioni di S\&S e delle soluzioni che gli hanno fatto ricavare la struttura a sezioni del disco secondo il tipo di pressione}

\section{Instabilità nelle regioni A}

	\emph{Terminologia di Shakura \& Sunyaev ripresa da altri autori per definire le zone con scattering Thompson e pressione di radiazione dominanti}\\
	

	\subsection{Analisi del problema}

		\emph{Articolo di Lightman \& Eardleay 1974 + Shakura \& Sunyaev 1976}

	\subsection{Ipotesi della doppia temperatura (sfera calda)}

		 \emph{Articolo di Shapiro, Lightman \& Eardley sul confronto con Cygn-X1, che però cambia la struttura del disco (togli otticamente spesso e cambia temperatura)}\\
	
	\subsection{Risultati delle simulazioni}

\newpage
\begin{thebibliography}{9}
	\bibitem{LightmanEardley1974} 
	A. P. Lightman, D. M. Eardley 
	"Black Holes in Binary Systems: Instability of Fisk Accretion"\\
	\textit{Astrop. Journal} 187, L1-L3, 1974 January 1
	
	\bibitem{MaozNutshell} 
	D. Maoz 
	"Astrophysics in a nutshell"\\
	\textit{Princeton University press} 2007
	
	\bibitem{ShakuraSunyaev1973} 
	N. I. Shakura, R. A. Sumyaev 
	"Black Holes in Binary Systems. Observational Appearance"\\
	\textit{Astron. \& Astrophys.} 24, 337-355 (1973)

	\bibitem{LightmanEardley1974} 
	S. L. Shapiro, A. P. Lightman, D. M. Eardley 
	"A Two-Temperature Disk Model for Cygnus X-1 Structure and Spectrum"\\
	\textit{Astrop. Journal} 187-199, 1976 February 15

	
	\bibitem{TeresiMolteniToscano2004} 
	V. Teresi, D. Molteni, E. Toscano 
	"SPH Simulations of Shakura-Sunyaev Instability at Intermediate Accretion Rates"\\
	\textit{Mon. Not. R. Astron. Soc.} 348, 361-367 (2004)
\end{thebibliography}

\end{document}
