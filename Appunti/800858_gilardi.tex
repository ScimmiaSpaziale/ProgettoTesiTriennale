\documentclass[a4paper]{article}
\usepackage[utf8]{inputenc}
\usepackage[italian]{babel}


\begin{document}
\textbf{{\Large Sintesi della relazione per la prova finale}}\\
\bigskip
{Corso di Laurea Triennale in Fisica}\\
\emph{Autore:} Riccardo Aurelio \textsc{Gilardi}\\
\bigskip
\textit{matricola}: 800858, \textit{email}: r.gilardi3@campus.unimib.it, \textit{tel}: 3491748105\\
\bigskip
\emph{Relatore:} Prof. Massimo \textsc{Dotti}\\
Sessione del 18 Marzo 2019

\subsection*{Studio della Stabilità dei Dischi di Accrescimento di Shakura \& Sunyaev}
Lo scopo di questa tesi è riassumere ed approfondire la stabilità delle regioni interne dei dischi di accrescimento, nel contesto del modello introdotto da Shakura e Sunyaev nel loro articolo del 1973 "\textit{Black Holes in binary Systems. Observational Appearance}", in riferimento alle osservazioni portate avanti per la prima volta da Lightman ed Eardley nel loro articolo del 1974 "\textit{Black Holes in binary Systems: Instability of Disk Accretion}".

Per mantenere una descrizione più semplice e meno dispersiva, ho deciso di analizzare, seguendo l'esempio di molti autori, un sistema formato da una stella ordinaria e un buco nero stellare non rotante.
Nell'analisi condotta non sono stati considerati effetti legati alla relatività generale o ai campi magnetici a cui il materiale in accrescimento potrebbe essere sottoposto.

Prima di parlare delle instabilità nei dischi di accrescimento, ho introdotto i concetti e le formule che descrivono un disco di accrescimento sottile, la fisica che ne governa lo stato stazionario e il meccanismo con cui si può arrivare alla sua formazione, cercando di utilizzare formule valide in generale, prima di introdurre le ipotesi di Shakura e Sunyaev.
Nella sezione finale della tesi ho invece cercato di mostrare analiticamente come il modello naif di Shakura e Sunyaev non permetta di costruire dischi di accrescimento stazionari, evidenziando come la loro soluzione sia necessariamente instabile nelle regioni del disco dominate da pressione di radiazione.

\subsection*{Stability Study of Shakura \& Sunyaev's Accretion Disks}

The aim of this work is to review and describe the theory behind the stability of accretion disks' inner region, within the model introduced by Shakura \& Sunyaev in their 1973 paper "\textit{Black Holes in binary Systems. Observational Appearance}". This study mainly refers to Lightman and Eardley's observations, which wew made for the first time in their 1974 paper "\textit{Black Holes in binary Systems: Instability of Disk Accretion}".

To maintain the description as simple and concise as possible, I decided to study a binary system formed by an ordinary star and a non rotating stellar black hole.
In the analysis I conducted general relativity effects and magnetic effect on the disk's matter haven't been considered.

I started describing the concepts and formulas that describe a thin accretion disk, the physics of its stationary state and the mechanism with which it forms, trying to use general results before introducing Shakura and Sunyaev's hypothesis.
I then tried to demonstrate analitically how the Shakura e \& Sunyaev's naif model leads to non-stationary and non-stable accretion disk. In particular I tried to show how their model is inherently unstable in the radiation pressure dominated regions of the disks.
\end{document}