\documentclass[a4paper]{article}
\usepackage[utf8]{inputenc}
\usepackage[italian]{babel}

\begin{document}
Studio della Stabilità dei Dischi di Accrescimento di Shakura & Sunyaev

Stability Study of Shakura & Sunyaev's Accretion Disks

Lo scopo di questa tesi è riassumere ed approfondire le teorie sulla stabilità delle regioni interne dei dischi di accrescimento, nel contesto del modello introdotto da Shakura e Sunyaev nel loro articolo del 1973 "Black Holes in binary Systems. Observational Appearance", in riferimento alle osservazioni portate avanti per la prima volta da Lightman ed Eardley nel loro articolo del 1974 "Black Holes in binary Systems: instabilityof Disk Accretion".
Per mantenere una descrizione più semplice e meno dispersiva, ho deciso di analizzare, seguendo l’esempio di molti autori, un sistema formato da una stella ordinaria e un buco nero stellare non rotante.
Nell’analisi che condurremo non vengono considerati effetti legati alla relatività generale o ai campi magnetici a cui il materiale in accrescimento potrebbe essere sottoposto.
Prima di parlare delle instabilità nei dischi di accrescimento, introdurrò i
concetti e le formule che descrivono un disco di accrescimento sottile, la fisica che ne governa lo stato stazionario e il meccanismo con cui si può arrivare alla sua formazione, cercando di utilizzare formule valide in generale, prima di introdurre le ipotesi di Shakura e Sunyaev.

The aim of this work is to review and describe the theory behind the stability of accretion disks' inner region, within the model introduced by Shakura & Sunyaev in their 1973 paper "Black Holes in binary Systems. Observational Appearance". This study refers primarily to the observations firstly made by Lightman and Eardley in their 1974 paper "Black Holes in binary Systems: instability of Disk Accretion".
To maintain the description as simple and concise as possible, I decided to study a binary system formed by an ordinary star and a non rotating stellar black hole.
In the analysis I conducted general relativity effects and magnetic effect on the disk's matter haven't been considered.
Before talking about the disk's instabilities, I started describing the concepts and formulas that describe a thin accretion disk, the physics of its stationary state and the mechanism with which it forms, trying to use general results before introducing Shakura and Sunyaev's hypothesis.
\end{document}
