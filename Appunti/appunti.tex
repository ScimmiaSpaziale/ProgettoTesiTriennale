\documentclass[a4paper]{article}
\usepackage[utf8]{inputenc}
%\usepackage[italian]{babel}
\usepackage{titling}
\usepackage{graphicx}
\usepackage{wrapfig}
\usepackage{float}
\usepackage{amsmath}
\usepackage{listings}
 \usepackage[table,xcdraw]{xcolor}

\newcommand{\subtitle}[1]{%
	\posttitle{%
		\par\end{center}
	\begin{center}\large#1\end{center}
	\vskip0.5em}%
}

\begin{document}

%opening
\title{Instabilità nelle Regioni Interne dei Dischi di Accrescimento a Regime Sub-Critico}
\subtitle{Appunti}
\author{Riccardo Aurelio Gilardi}
\maketitle

\newpage
\tableofcontents
\newpage

\section{Accrescimento nei sistemi binari (introduzione)}
	\subsection{Momento angolare}
	\subsection{Modello $\alpha$-disco}
	
		\emph{Data la natura rivoluzionaria di molte delle sue ipotesi, seguirò più o meno un processo storico, parlando prima del modello secondo S\&S, magari citando anche Pringle \& Reese e Norikov \& Thorne (non riesco a trovare quel foglio, btw) confrontando le sue ipotesi (disco sottile, otticamente spesso etc.) con le possibili alternative, per giustificarne le scelte.}

\section{Struttura del disco secondo Shakura \& Sunyaev}

	\emph{Devo spiegare le equazioni di S\&S e delle soluzioni che gli hanno fatto ricavare la struttura a sezioni del disco secondo il tipo di pressione}

\section{Instabilità nelle regioni A}

	\emph{Terminologia di Shakura \& Sunyaev ripresa da altri autori per definire le zone con scattering Thompson e pressione di radiazione dominanti nel loro modello.\\Non posso parlare di "zone dominate da pressione di radiazione" perché anche se questa è la descrizione del primo modello, poiché l'instabilità potrebbe non essere reale, quanto piuttosto un difetto del modello, che viene appunto modificato nell'articolo Shapiro, Lightman \& Eardley del 1974}\\
	
	\subsection{Analisi del problema}

		\emph{Articolo di Lightman \& Eardleay 1974 + Shakura \& Sunyaev 1976}

	\subsection{Primi studi: Ipotesi della doppia temperatura (sfera calda)}

		\emph{Articolo di Shapiro, Lightman \& Eardley sul confronto con Cygn-X1, che però cambia la struttura del disco (togli otticamente spesso e cambia temperatura), da controbattere con le correzioni di Shakura e Sunyaev 1976}\\
	
	\subsection{Ipotesi di altri autori (Da Pringle (1981))}
	
		\emph{Forse è superfluo, dato che sono tra il primo problema e le soluzioni moderne...}
		
	\subsection{Magneto-stabilità rotazionale}
		\emph{Dal Frank King Raine, parlo delle descrizioni moderne (cercare articoli recenti a riguardo?)}

\newpage
\begin{thebibliography}{9}
	\bibitem{FrankKingRaineAccretionPower} 
	J. Frank, A. King, D. Raine
	"Accretion Power in Astrophysics"\\
	\textit{Cambridge University Press}, 2002 (III ed.)\\
	
	\bibitem{King2012} 
	A. King 
	"Accretion Disc Theory since Shakura and Sunyaev"\\
	\textit{arXiv}: 1201.2060v1\\
	\textit{to appear in proceedings of 'The Golden Age of Cataclysmic Variables', Memorie Società Astronomica Italiana, 2012 (F. Giovannelli and L. Sabau-Graziati eds.)}

	\bibitem{LightmanEardley1974} 
	A. P. Lightman, D. M. Eardley 
	"Black Holes in Binary Systems: Instability of Fisk Accretion"\\
	\textit{Astrop. Journal} 187, L1-L3, 1974 January 1
	
	\bibitem{MaozNutshell} 
	D. Maoz 
	"Astrophysics in a nutshell"\\
	\textit{Princeton University press}, 2007
		
	\bibitem{Pringle1981} 
	J. E. Pringle 
	"Accretion Discs in Astrophysics"\\
	\textit{Ann. Rev. Astron. Astrphys.} 1981, 19:137-62
		
	\bibitem{PringleReese1972} 
	J. E. Pringle, M. J. Rees
	"Accretion Discs Model for Compact X-Ray Sources"\\
	\textit{Astron. \& Astrphys.} 21, 1-9 (1972)
	
	\bibitem{PringleReesePacholczyk1973} 
	J. E. Pringle, M. J. Rees, A. G. Pacholczyk
	"Accretion onto Massive Black Holes"\\
	\textit{Astron. \& Astrphys.} 29, 179-184 (1973)
		
	\bibitem{ShakuraSunyaev1973}
	N. I. Shakura, R. A. Sumyaev 
	"Black Holes in Binary Systems. Observational Appearance"\\
	\textit{Astron. \& Astrophys.} 24, 337-355 (1973)

	\bibitem{ShakuraSunyaev1976}
	N. I. Shakura, R. A. Sumyaev 
	"A Theory of the Instability of Disk Accretion on to Black Holes and the Variability of Binary X-Ray Sources, Galactic Nuclei and Quasars"\\
	\textit{Mon. Not. R. astr. Soc.} (1976) 175, 613-632
	
	\bibitem{ShakuraZeldovich2018}
	N. I. Shakura
	"Ya. B. Zeldovich and foundation of the accretion theory"\\
	\textit{arXiv}: 1809.1137v1

	\bibitem{LightmanEardley1974} 
	S. L. Shapiro, A. P. Lightman, D. M. Eardley 
	"A Two-Temperature Disk Model for Cygnus X-1 Structure and Spectrum"\\
	\textit{Astrop. Journal} 187-199, 1976 February 15
	
	\bibitem{TaamLin1984} 
	R. E. Taam, D. N. C. Lin 
	"The Evolution of the Inner Regions of Viscous Accretion Disks Surrounding Neutron Stars"\\
	\textit{Astrop. Journal} 287, 761-768 1984 December 15

	\bibitem{TeresiMolteniToscano2004} 
	V. Teresi, D. Molteni, E. Toscano 
	"SPH Simulations of Shakura-Sunyaev Instability at Intermediate Accretion Rates"\\
	\textit{Mon. Not. R. Astron. Soc.} 348, 361-367 (2004)
\end{thebibliography}

\end{document}
